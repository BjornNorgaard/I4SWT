\section{Integrationstest}
Integrationstests udføres for at checke om de enkelte software moduler passer sammen. Helt konkret drejer det sig om at finde de komponenter der forårsager \textit{Intercomponent failures}. De fleste interoperability fejl findes ikke med komponent scope testing Det er vigtigt at teste dette før der testes i \textit{system scope}.

I OO programmering kan integrationstest forgå løbende.

\subsection{Intregrationstest strategi}

\begin{itemize}
	\item Hvilke komponenter er i fokus?
	\item I hvilken rækkefølge skal komponenterne testes?
	\item Hvilken test design teknik skal bruge?
\end{itemize}

Før man begynder at integrationsteste er der nogle ting som man bør have opfyldt:
\begin{itemize}
	\item Unittests skal være færdige. Dette gælder for samtlige metoder!
	\item Systemets dependancy tree skal være kendt! jf. ATM opgaven.
	\item Integrationstesten er planlagt.
\end{itemize}

\subsection{Typer af integrationstest}
Der findes flere typer af tilgange til at integrationsteste.

\subsubsection{Dependancy trees}
Dependancy trees er en måde at visualisere afhængighederne i et projekt. Det er vigtigt at have et overblik over sine dependancies når man skal planlægge integrationstest. På figur~\ref{fig:dependancyATM} vises et uddrag fra ATM opgavens dependancy tree.

\subsubsection{Drivers}
En driver er en klasse, eller et program der påtrykker testcases på den komponenten under test.

\begin{figure}[H]
\centering
\includegraphics[width=0.8\linewidth]{figs/dependancyATM.PNG}
\caption{Uddrag fra ATM's dependancy model}
\label{fig:dependancyATM}
\end{figure}

Dependancy træet er opbygget så høj niveau modulerne ligger i den ene ende og lav niveau modulerne i den anden ende.

%%%%%%%%%%%%%%%%%%%%%%%%%%%%%%%%%%%%%%%%%%%%%%%%%%%%%%%%%%%%%%%%%%%%%%%%%%%%%%%%%%%%%%%%%%%%%%%%%%

\subsubsection{Big Bang testing}
I en \textit{Big Bang} integrationstest, sammensættes alle software komponenter, så det komplette system dannes. Herefter testes systemet.

\paragraph{Hvorfor Big Bang testing?}
\begin{itemize}
	\item Man slipper for at planlægge sin test
	\item Nemt at bruge i mindre systemer
\end{itemize}

\paragraph{Hvorfor ikke Big Bang testing?}
Der er en del ulemper ved denne integratoinstest model.

\begin{itemize}
	\item Man finder først fejlene når alle komponenter sættes sammen.
	\item Det er svært at isolere de fundne fejl.
	\item Der er stor sansynlighed for at misse kritiske fejl, som senere kan vise sig. \todo{hvorfor det?}
	\item Det er svært at dække alle test scenarier uden at misse enkelte.
	\item Bugfixing er last-minute og er derfor ofte af dårlig kvalitet.
\end{itemize}

%%%%%%%%%%%%%%%%%%%%%%%%%%%%%%%%%%%%%%%%%%%%%%%%%%%%%%%%%%%%%%%%%%%%%%%%%%%%%%%%%%%%%%%%%%%%%%%%%%

\subsubsection{Bottom-Up testing}

\begin{itemize}
	\item Starter i bunden af dependancy træet - nederste lag testes først.
\end{itemize}

\begin{figure}
\centering
\includegraphics[width=0.7\linewidth]{figs/bottomUpFlow.PNG}
\caption{Illustration af Bottom-Up testing.}
\label{fig:bottomUpFlow}
\end{figure}

\paragraph{Hvorfor Bottom-Up testing?}

\paragraph{Hvorfor ikke Bottom-Up testing?}

%%%%%%%%%%%%%%%%%%%%%%%%%%%%%%%%%%%%%%%%%%%%%%%%%%%%%%%%%%%%%%%%%%%%%%%%%%%%%%%%%%%%%%%%%%%%%%%%%%

\subsubsection{Top-Down testing}

\begin{itemize}
	\item Starter i toppen af dependancy træet - øverste lag testes først.
\end{itemize}

\begin{figure}
\centering
\includegraphics[width=0.7\linewidth]{figs/topDown.PNG}
\caption{Illustration af Top-Down testing.}
\label{fig:topDown}
\end{figure}

\paragraph{Hvorfor Top-Down testing?}

\paragraph{Hvorfor ikke Top-Down testing?}

%%%%%%%%%%%%%%%%%%%%%%%%%%%%%%%%%%%%%%%%%%%%%%%%%%%%%%%%%%%%%%%%%%%%%%%%%%%%%%%%%%%%%%%%%%%%%%%%%%

\subsubsection{Collaboration testing}

\begin{itemize}
	\item I collaboration test tager man en "gren" ad gangen i dependency træet.
\end{itemize}

\begin{figure}
\centering
\includegraphics[width=0.7\linewidth]{figs/collaborationTesting.PNG}
\caption{Illustration af Collaboration testing.}
\label{fig:collaborationTesting}
\end{figure}

\paragraph{Hvorfor Collaboration testing?}

\paragraph{Hvorfor ikke Collaborationp testing?}

%%%%%%%%%%%%%%%%%%%%%%%%%%%%%%%%%%%%%%%%%%%%%%%%%%%%%%%%%%%%%%%%%%%%%%%%%%%%%%%%%%%%%%%%%%%%%%%%%%

\subsubsection{Sandwich testing}

\begin{figure}
\centering
\includegraphics[width=0.7\linewidth]{figs/sandwich.PNG}
\caption{Illustration af Sandwich testing.}
\label{fig:sandwich}
\end{figure}

\paragraph{Hvorfor Sandwich testing?}

\paragraph{Hvorfor ikke Sandwich testing?}