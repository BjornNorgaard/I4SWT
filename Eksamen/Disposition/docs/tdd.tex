\section{Test-Driven Development (TDD)}

TDD er en udviklignsmetode hvor man skriver sine test \textbf{først}, ser denne test fejle. Derefter implementeres den mindst mulige funktionalitet til at få testen til gå godt.

\begin{figure}[h]
\centering
\includegraphics[width=\linewidth]{figs/tdd}
\caption{Red-Green-Refactor mamtra med TDD.}
\label{fig:tdd}
\end{figure}

\subsection{Feature list}
En \textit{feature list} indeholder (pudsigt nok) features, som ønskes implementeret. Herefter udvikler man bare én feature af gangen. \textit{Feature listen} skal indeholde en beskrivelse samt et sæt af test for den pågældende feature.

\subsection{Red-Green-Refactor}
Bruges til at beskrive selve arbejdsmetoden som anvendes ved TDD.

\begin{enumerate}
	\item Skriv en test som skal fejle\footnote{Hvis testen ikke fejler første gang, er der noget galt.}.
	\item Koden skrives så testen præcis bestås.
	\item Koden refaktueres.
\end{enumerate}

Fordelen ved dette skulle gerne være at man er tvunget til at skrive \textbf{blackbox-test}, at koden bliver meget testbar og unødvendig koden holdes på et minumum.

\begin{figure}[H]
\centering
\includegraphics[width=0.7\linewidth]{figs/redgreen}
\caption{Red-Green-Refactor. Fejl, Implementer, refactor.}
\label{fig:redgreen}
\end{figure}

\subsection{Fordele og Ulemper}

\begin{itemize}
	
	\item \textbf{Fordele}	
	\begin{itemize}
		\item Koden bliver forklaret gennem test, jf. naming-convention.
		\item Man overvejer nødvendigheder i stedet for implementering.
		\item Koden bliver testbar og gennemtestet.
	\end{itemize}
	
	\item \textbf{Ulemper}	
	\begin{itemize}
		\item Forlænget udviklingstid - skulle gerne lede til besparet tid på vedligeholdelse mm.
		\item Kræver stor selvdisiplin - man skal ikke ''bare lige'' implementerer noget før det er nødvendigt.
	\end{itemize}
		
\end{itemize}
































