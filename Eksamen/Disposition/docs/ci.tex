\section{Continuous Integration}

CI er en practice hvor udviklere integrere deres arbejde ofte. Ifølge Martin Fowler, gerne mininmum dagligt.
Hver integration verificeres med en automatisk build og test. På denne måde findes integrationsfejl hurtigt.

\begin{figure}[H]
\centering
\includegraphics[width=0.7\linewidth]{figs/ciArch.PNG}
\caption{Modellen for Continous Integration i et projekt}
\label{fig:ciArch}
\end{figure}

\subsection{CI funktionalitet}
På figur~\ref{fig:ciTriggerSeq} ses trigger sekvensen for et automatisk \textit{\textbf{build}} og \textit{\textbf{run}}.

\begin{figure}[H]
\centering
\includegraphics[width=0.8\linewidth]{figs/ciTriggerSeq.PNG}
\caption{Trigger sekvens for Continous Integration med Git repository.}
\label{fig:ciTriggerSeq}
\end{figure}

Fordelene ved Continous Integration er:

\begin{itemize}
	\item Hele projektet testes samlet.
	\item Test sker uden for miljøet - Ved store projekter spares tid.
	\item Der pushes deployable builds \todo{er det et krav at man kun pusher deployable builds i jenkins?}
	\item Der kan pushes ofte, så der findes fejl før der skrives meget kode.
	\item Code metrics/quality er altid opdateret (hvis CI systemet understøtter det).
\end{itemize}

\subsubsection{Graded builds}

\begin{itemize}
	\item Continous build.
	\item Nighly build.
\end{itemize}

\subsection{CI Systemer}
I SWT arbejdes der med Jenkins serveren. Udover at bygge et software projekt, kører den også tilhørende tests, hvorefter den kan sende en mail om resultatet.
Jenkins mantra: build, test, result, mail. \todo{Er det en ting det her?}

Derudover er der en stor del funktionalitet i Jenkins. Herunder Code Coverage og Metrics.

\todo{lav eksempel med ATM opgaven}

\todo{skriv om Travis og evt. forskelle mellem Jenkins og Travis}